\chapter{Método}
\label{chapter:method}

Este capítulo descreve a elaboração de classificadores de sentimento de tweets
usando técnicas de processamento de linguagem natural e ciência de redes.
Serão detalhados as etapas de coleta de base de dados, as figuras de mérito
utilizadas nas avaliações dos modelos e o processo de treinamento dos
algoritmos.

\section{Bases de Dados}

% TODO: preencher os valores
Serão utilizadas duas bases de dados nesse trabalho, uma de treinamento e uma de
teste.
Tendo objetivo de diminuir o custo de desenvolvimento do classificador será
adotado o método de anotação distante descrito por~\citet{go09} para formação da
base de dados de treinamento.
Esta é formada por 10 milhões de tweets em português coletados entre 2018 e 2020.

Sobre esses foram selecionados manualmente emoticons positivos e negativos
dentre os mais frequentes. A Tabela~\ref{tab:emoticons} mostra os emoticons
escolhidos.
Mensagens que possuírem um ou mais dos emoticons selecionados serão
categorizados com a classe correspondente na base de treinamento.
Para remover possíveis ambiguidades foram removidos da base os textos que
possuiram mais do que uma categoria.
Além disso, os emoticons utilizados na supervisão distante foram removidos dos
textos para evitar a introdução de vies no classificador.
Desta a base de treinamento é composta de YYYY tweets, sendo YYY positivos e ZZZ
negativos.

% FIX
\begin{table}[h]
    \begin{center}
        \begin{tabular}{| l | c |}
        \hline
        \textbf{Classe} & \textbf{Emoticons} \\ \hline
        Positiva & :) \\ \hline
        Negativa & :) \\ \hline
        \end{tabular}
        \caption{Emoticons selecionados para aplicação de supervisão distante.}
        \label{tab:emoticons}
    \end{center}
\end{table}

A base de testes é composta por XYZ tweets anotados manualmente.
Os tweets dessa base também foram amostrados entre 2018 e 2020 e não pertencem a
base de treinamento.
Dentre o conjunto total YYY mensagens são positivas e WWWW são negativas.
A anotação foi feita utilizando a ferramenta
Doccano~\footnote{https://github.com/doccano/doccano}.

\section{Figuras de Mérito}

A avaliação dos algoritmos aplicados nestas bases será feita utilizando as figuras
de mérito descritas nessa seção.

Levando em consideração o desbalanceamento de classes das bases de dados não é
adequado a aplicação da acurácia pois esta favoreceria a classe majoritariamente
presente nas bases.
Como alternativa será adotada a área sob a curva ROC (\textit{receiver operating
characteristic}).
A curva ROC apresenta a relação entre probabilidade de detecção e falso alarme
para diferentes patamares de decisão de classificadores binários.
A área sob a curva ROC é uma medida que agrupa esse conjunto de taxas em um
número único, que representa o modelo em todos possíveis pontos de operação.

Como o objetivo abordado não apresenta um ponto de operação pré-definido, esse
também pode ser definido pela otimização de uma figura de mérito.
Desta forma, A seleção do limiar do classificador treinado será feita a partir do
índice SP~\cite{ciodaro12}.
Este índice consiste na média geométrica entre os pontos de média geométrica e
média aritmética entre a probabilidade de detecção da classe positiva $P_c$ e a
probabilidade de não obtenção de falso alarme $P_{nc}$, como mostra a
Equação~\ref{eq:sp}.

\begin{equation} \label{eq:sp}
    SP = \sqrt{\sqrt{P_c P_{nc}} \left(\frac{P_c + P_{nc}}{2}\right)}
\end{equation}

\section{Desenvolvimento}
% usar graphSAGE como aggregador de embedding de texto do usuario? ler pagina 2
% de hamilton17
