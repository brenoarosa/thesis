\begin{abstract}

Nos últimos anos, as redes sociais se tornaram um dos principais meios de
comunicação e com isso, houve um aumento da influência que exercem sobre os
usuários.
Por esse motivo, as mensagens que trafegam por elas passam a ter importância
para as mais diversas finalidades como, por exemplo, a avaliação de produtos e
eventos.
Dentre as possíveis análises, a mineração de opinião é uma das operações com mais
aplicações diretas.
Nesse sentido, ferramentas de processamento de linguagem natural e de redes
complexas são capazes de auxiliar a geração destas análises.
Entretanto, o desempenho dessas técnicas, em geral, depende da existência e do
volume de bases de treinamento anotadas manualmente, dificultando assim a
utilização das mesmas.
O presente trabalho aplica métodos de geração de bases de treinamento
automatizadas para contornar esse obstaculo e gerar classificadores de análise
de sentimento.
São avaliados diferentes modelos de classificação textual, tanto lineares,
como Naïve Bayes e SVM, quanto por \textit{Deep Learning}, como redes
convolucionais e redes recorrentes.
Também são analisadas técnicas de redes complexas para caracterização dos
usuários das redes, abordando assim diferentes aspectos das informações
fornecidas por estas mídias.

\end{abstract}
