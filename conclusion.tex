\chapter{Conclusões}
\label{chapter:conclusion}

Esse trabalho explorou o uso de métodos de aprendizado de máquina multimodais
para a análise de sentimento de publicações de redes sociais.

Ao longo das últimas duas décadas observamos uma crescente relevância das mídias
sociais.
Esse crescimento também refletiu no aumento de volume de mensagens que
trafegam por essas redes.
Esses fatores ressaltam a necessidade de ferramentas capazes de extrair
informação a partir dessa grande massa de dados, possibilitando entender o
sentimento dos usuários das redes sobre um tópico, evento ou produto.
Em paralelo, observou-se a evolução das técnicas de processamento de texto por
aprendizado de máquina, se mostrando poderosos métodos para a análise de
sentimento e resultados promissores aplicados a outras categorias textuais.

Entretanto, as publicações provindas de redes sociais são constituídas de vários
elementos além dos textos, como dados do usuário, imagens, localização, horário,
etc.
Essa multimodalidade é um dos principais componentes que diferenciam as mídias
sociais dentre os outros meios de comunicação.
Este trabalho também estudou a aplicação técnicas de representação de grafos para
capturar a informação da rede formada por usuários e seus impactos na análise de
sentimento.

Visando a facilidade de reprodução da classificação em qualquer base de dados
adotamos a técnica de supervisão distante.
Este método permite o uso de modelos supervisionados de classificação sem o
custo necessário para anotar grandes bases de treinamento.
Foram realizados experimentos com $9,2$ milhões de \textit{tweets} coletados
entre 2018 e 2019.
A partir dos experimentos observamos que apesar do sucesso de técnicas baseadas
em redes neurais em outras modalidades de texto, estas técnicas não se mostraram
tão capazes neste conjunto de dados.
Os modelos lineares, que já são tradicionalmente aplicados a processamento de
linguagem natural ao longo das últimas décadas, obtiveram resultados superiores aos
modelos neurais.
Foram analisados nesse estudo modelos de \textit{Naive Bayes}, SVM, redes
neurais convolucionais e LSTM.
Também observamos que a representação de usuários a partir do grafo de seus
\textit{retweets} não foi capaz de superar o melhor modelo puramente textual.
Foram avaliadas as representações por \textit{Locally Linear Embedding},
\textit{Node2Vec} e rede convolucional de grafos.
Analisamos portanto que apesar da importância de elementos não-textuais as
informações extraídas pelas técnicas avaliadas de representação não
supervisionadas de vértices não agregaram informações relevantes para a análise
de sentimento.


%%% secao: trabalhos futuros
