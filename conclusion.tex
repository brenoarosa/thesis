\chapter{Conclusões}
\label{chapter:conclusion}

Esse trabalho explorou o uso de métodos de aprendizado de máquina multimodais
para a análise de sentimento de publicações de redes sociais.

Ao longo das últimas duas décadas observamos uma crescente relevância das mídias
sociais.
Esse crescimento também refletiu no aumento de volume de mensagens que
trafegam por essas redes.
Esses fatores ressaltam a necessidade de ferramentas capazes de extrair
informação a partir dessa grande massa de dados, possibilitando entender o
sentimento dos usuários das redes sobre um tópico, evento ou produto.
Em paralelo, observou-se a evolução das técnicas de processamento de texto por
aprendizado de máquina, se mostrando poderosos métodos para a análise de
sentimento e resultados promissores aplicados a outras categorias textuais.

Entretanto, as publicações provindas de redes sociais são constituídas de vários
elementos além dos textos, como dados do usuário, imagens, localização, horário,
etc.
Este trabalho também estudou a aplicação técnicas de representação de grafos para
capturar a informação da rede formada por usuários.
Essa multimodalidade 

% reforcar importancia do trabalho e sua utilidade pratica

% ressaltar que consegue usar o volume e informacoes adicionais alem do texto

% falar das tecnicas tentadas e resumo dos resultados


%%% secao: trabalhos futuros
