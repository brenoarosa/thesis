\chapter{Modelos de Redes Complexas}
\label{chapter:networks}

Grafos são estruturas de dados que codificam relações, e estão presentes em uma
grande variedade de cenários.
Estes são compostos por nós que representam elementos, quaisquer que sejam, e
arestas que são as relações entre os elementos na rede.
Se tratando de redes sociais grafos são especialmente importantes dado que
a relação entre os usuários é a principal componente desses serviços.
Ao longo das ultimas décadas o estudo de modelos aplicados aos grafos, foi se
tornando cada vez mais relevante.
Esses modelos visam capturar diversas características possíveis das redes como a
evolução do grafo no tempo, a propagação de epidemias dentro da rede,
recomendação de arestas, a predição de classes de um dado nó ou aresta, entre outros.

Neste capitulo serão estudados modelos de representações de nós.
Assim como no caso de documentos textuais, estes modelos, em geral, visam
codificar a informação de um nó em um vetor de baixa dimensionalidade,
capturando tanto seus atributos individuais quanto os decorrentes da estrutura
de conexões do mesmo.
Este objetivo se adequa ao presente trabalho pois permite que seja treinado um
classificador único que utilize tanto a informação capturada pela linguagem quanto
da a provinda da rede.

% FIGURA: com subfiguras representado cada uma das tarefas citadas
% - classificacao de nós: grafos com nós em 2 cores
% - classificação de grafos: multiplos grafos, divididos em 2 cores...

% pegar exemplo de figura de artigo de classificacao de nos
% de 2 grafos egocentricos de classes distintas

% grande referencia em graph embeddings
% A Comprehensive Survey of Graph Embedding: Problems, Techniques and Applications
% outra opcao: Graph convolutional networks: a comprehensive review
