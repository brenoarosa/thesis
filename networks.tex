\chapter{Modelos de Redes Complexas}
\label{chapter:networks}

Grafos são estruturas de dados que codificam relações, e estão presentes em uma
grande variedade de cenários.
Estes são compostos por nós que representam elementos, quaisquer que sejam, e
arestas que são as relações entre os elementos na rede.
Se tratando de redes sociais grafos são especialmente importantes dado que
a relação entre os usuários é a principal componente desses serviços.
Ao longo das ultimas décadas o estudo de modelos aplicados aos grafos, foi se
tornando cada vez mais relevante.
Esses modelos visam capturar diversas características possíveis das redes como a
evolução do grafo no tempo, a propagação de epidemias dentro da rede,
recomendação de arestas, a predição de classes de um dado nó ou aresta, entre outros.

Neste capitulo serão estudados modelos de representações de nós.
Assim como no caso de documentos textuais, estes modelos, em geral, visam
codificar a informação de um nó em um vetor de baixa dimensionalidade,
capturando tanto seus atributos individuais quanto os decorrentes da estrutura
de conexões do mesmo.
Este objetivo se adequa ao presente trabalho pois permite que seja treinado um
classificador único que utilize tanto a informação capturada pela linguagem quanto
da a provinda da rede.

% FIGURA: com subfiguras representado cada uma das tarefas citadas
% - classificacao de nós: grafos com nós em 2 cores
% - classificação de grafos: multiplos grafos, divididos em 2 cores...

% pegar exemplo de figura de artigo de classificacao de nos
% de 2 grafos egocentricos de classes distintas

% grande referencia em graph embeddings
% A Comprehensive Survey of Graph Embedding: Problems, Techniques and Applications
% Graph Embedding Techniques, Applications, and Performance: A Survey

\section{Modelos Baseados em Fatoração Matricial}

A primeira modalidade de técnicas de representação de nós a ser desenvolvida foi
a por fatoração matricial.
Essa consiste em utilizar o resultado da fatoração de uma das matrizes
características da rede como representação dos nós, dentro os exemplos de
matrizes dos grafos temos: a matriz de adjacência, a matriz Laplaciana, a matriz
de centralidade de Katz, entre outras~\cite{goyal18}.
Nos casos que a matriz é positiva e semi-definida, como a matriz Laplaciana, é
possível aplicar a decomposição em autovalores, nos outros casos, em geral, são
utilizados métodos iterativos visando minimizar uma função custo.

\subsubsection{Locally Linear Embedding (LLE)}

\citet{roweis00} desenvolveram esse algoritmo que tem como premissa que um nó é
uma combinação linear de seus vizinhos.
Portanto tendo a matriz de adjacencia $W_{ij}$ e o vetor de representação de um
nó como $Y_i$, definimos o mesmo como:

\begin{equation}
    Y_i \approx \sum_j{W_{ij}Y_j}
\end{equation}

A função custo a ser minimizada para atingir é a que minimiza o erro de
representação do somatório de todos os nós, sendo assim a função custo $\phi(Y)$
é dada por:

\begin{equation}
    \phi(Y) = \sum_i{|Y_i - \sum_j{W_{ij}Y_j}|^2}
\end{equation}

Sendo necessário respeitar a restrição $\frac{1}{N}Y^TY = 1$ para XXXX.
Esta técnica obtém uma representação de nós que preserva as distância de
primeira ordem da rede, ou seja, vizinhos diretos no grafo original vão obter
representações próximas.

\subsubsection{Automapas Laplacianos}

Esta técnica desenvolvida por \citet{belkin02} também tem como objetivo obter
representações próximas entre vizinhos, desta vez inspirada na equação de fluxo
de calor, usando a matriz laplaciana do grafo.
Sua função custo é dada pela equação~\ref{eq:laplacian_eigenmaps} na qual $L$ é
a laplaciana do grafo.

\begin{equation} \label{eq:laplacian_eigenmaps}
\begin{aligned}
    \phi(Y) &= \frac{1}{2} \sum_{ij}{|Y_i - Y_j|^2 W_{ij}} \\
            &= tr(Y^TLY)
\end{aligned}
\end{equation}

\subsubsection{HOPE}

Os algoritmos apresentados anteriormente são efetivos para manter a distância
entre vizinhos de um grafo não-direcionado.
Entretanto, grafos direcionados reais que em grande parte das vezes tem sua
distribuição de grau seguindo uma lei de potência, como é o caso dos grafos
formados em redes sociais, as propriedades de um par de nós não são simétricas.
Para obter representações que respeitem essa assimetria \citet{ou16}
desenvolveram o algoritmo \textit{High-Order Proximity preserved Embedding}
(HOPE).
A chave deste algoritmo é utilizar métricas de proximidade de ordem superior em
sua função custo, preservando assim as propriedades da assimetria.
Qualquer medida de proximidade pode ser aplicada neste algoritmo, \citet{ou16}
em seu estudo testaram medidas como a proximidade de Katz e Pagerank
personalisado, mostrando que em varios casos essas medidas possuem aproximações
que diminuem o custo computacional da otimização.

HOPE representa cada vértice em 2 vetores, um de entrada e um de saída, o qual
denominaremos respectivamente $Y^s$ e $Y^t$.
A função custo é dada pela equação~\ref{eq:hope} na qual $S$ representa a matriz
de proximidade entre os nós dada por qualquer uma das medidas escolhidas.

\begin{equation} \label{eq:hope}
    \phi = \vert\vert S - Y^s {Y^t}^{T} \vert\vert^2_F
\end{equation}

\section{Modelos Baseados em Passeios Aleatórios}

Outra estratégia amplamente adotada são as baseadas em Passeios Aleatórios.
O Passeio Aleatório é um processo estocástico que consiste em se selecionar um
vértice inicial no grafo e completar uma sequência de passos aleatórios.
Sua aplicação para obtenção de representações considera a sequência de nós de
múltiplas realizações do Passeio Aleatório como uma função de proximidade entre
os nós.
Por ser iterativo e não necessitar o calculo de nenhuma matriz completa do
grafo, esse método apresenta vantagem em custo operacional para redes grandes.
Além disso, a iteratividade do método também permite que novos elementos sejam
adicionados ao grafo sem requerer um treinamento completo do modelo,
característica essencial para seleção de algoritmos em diversas aplicações.
A seguir serão apresentados os principais modelos deste grupo.

\subsubsection{Deepwalk}

Deepwalk~\cite{perozzi14} se inspira em modelos de representações de palavras
utilizando como entrada sequências de iterações de Passeios Aleatórios truncadas
em poucos passos.
Os autores observaram que assim como a distribuição de ocorrência de palavras de
uma lingua segue uma lei de potência, mesmo comportamento de boa parte das redes
formadas no mundo real, como a de conexão entre pessoas.
Desta forma, adaptaram algoritmos previamente aplicados para predição de
palavras para obter as representações dos vértices.

O modelo proposto é muito similar a variação \textit{skipgram} do
Word2Vec~\ref{sec:w2v}, os passeios aleatórios truncados formam sequências de
tamanho $2k + 1$, o treinamento do algoritmo é feito de maneira a maximizar a
probabilidade de predição do contexto a partir do vértice na posição central do
passeio.
A formula~\ref{eq:deepwalk} formaliza o problema citado dado que $v_i$ é o
vértice central de um dado passeio e $\Phi$ é o mapeamento de um vetor em sua
representação, que queremos encontrar.

\begin{equation} \label{eq:deepwalk}
    \operatorname{max} P(\{v_{i-k},...,v_{i-1},v_{i+1},...,v_{i+k}\}|\Phi(v_i))
\end{equation}

O processo é feito de forma iterativa em que cada realização do Passeio
Aleatório compões uma entrada da algoritmo.
A otimização é dada por gradiende decentente da função custo $\phi$ que maximiza
a probabilidade~\ref{eq:deepwalk}, esta é apresentada na
equação~\ref{eq:deepwalk_cost}.
Outras adaptações propostas posteriormente ao Word2Vec para diminuir o custo
computacional de treinamento também podem ser adotadas no Deepwalk, como
\textit{softmax} hierarquico e \textit{negative sampling}.

\begin{equation} \label{eq:deepwalk_cost}
    \phi = -\log P(\{v_{i-k},...,v_{i-1},v_{i+1},...,v_{i+k}\}|\Phi(v_i))
\end{equation}

\subsubsection{node2vec}

O node2vec~\cite{grover16} também é um algoritmo inspirado no Word2Vec.
Seu processo de otimização é idêntico ao apresentado pelo Deepwalk, também
aplicado sobre sequências de passeios aleatórios de tamanho fixo.
A diferença entre ambos está na estratégia de passeio aleatório, que além de
capturar a vizinhança de um dado vértice visa também capturar a posição do mesmo
na estrutura ao seu redor.
Enquanto a vizinhança de um nó compõe uma parte de sua informação, o
posicionamento do mesmo em sua comunidade também é relevante.
Dentro de uma rede, um nó central de uma comunidade pode apresentar
características mais semelhantes ao ponto central de outra comunidade
do que a algum vértice vizinho com poucas conexões.

A proposta desenvolvida por \citet{grover16} consiste em um Passeio Aleatório
enviesado que parametrizam quão relevante são cada um desses dois comportamentos.
Ao se escolher parâmetros que priorizem a vizinhança do vértica, a amostragem do
Passeio Aleatório enviesado passa a se assemelhar a uma busca em profundidade,
por outro lado, focando-se na estrutura formada pelas conexões do nó obtemos uma
amostragem semelhante a busca em largura.

No Passeio Aleatório não enviesado, a probabilidade de transição $\pi_{uv}$ do
nó $u$ para o nó $v$ é dada por $\pi_{uv} = \frac{w_{uv}}{\sum{w_u}}$, no qual
$w_{uv}$ é o peso da aresta entre $u$ e $v$ e $\sum{w_u}$ é o somatório dos das
arestas de $u$, neste caso operando como regularizador da probabilidade.
Para atingir seu objetivo, node2vec multiplica a probabilidade de transição
original do Passeio Aleatório por um fator $\alpha(s, v)$ parametrizado pelos
parametros $p$ e $q$, como mostra a equação~\ref{eq:node2vec_alpha}, na qual
$s$ representa o vértice em que o Passeio Aleatório amostrou no passo anterior e
$d_{sv}$ representa a distância mínima entre o par de vértices $s$ e $v$.
Esses parâmetros controlam quão rápido o Passeio Aleatório explora a rede.

\begin{equation} \label{eq:node2vec_alpha}
    \alpha(s, v) =
    \begin{cases}
        \frac{1}{p} ,& \text{se } d_{sv} = 0\\
        1           ,& \text{se } d_{sv} = 1\\
        \frac{1}{q} ,& \text{se } d_{sv} = 2
    \end{cases}
\end{equation}

O parâmetro $p$ é chamado de parâmetro de retorno, a escolha de valores altos
para esse fator faz com que o Passeio Aleatório evite reamostrar vértices
recem amostrados, enquando valores baixos forçam o algoritmo a manter a
amostragem localizada em uma distância pequena de seu ponto de partida.
Por sua vez, o parâmetro $q$ regula a probabilidade de se amostrar nós distantes
do vértice de partida.
Valores baixos de $q$ resultam em uma amostragem semelhante a busca em
profundidade enquanto valores altos levam a um padrão de busca em largura.
Resumidamente, a probabilidade de transição do passeio aleatório proposto por
node2vec é dada pela equação~\ref{eq:node2vec_pi}.

\begin{equation} \label{eq:node2vec_pi}
    \pi(s, u, v) = \frac{w_{uv} \alpha(s,v)}{\sum_{x} w_{ux} \alpha(s,x)}
\end{equation}

Os autores mostram que a adoção dessa estratégia de amostragem resultou em
ganhos significativos de performance do algoritmo para tarefas de classificação
de vértices.
Apesar da maior complexidade, o algoritmo apresenta aumento linear do custo
computacional com relação ao número de nós da rede, possibilitando sua aplicação
em redes de grande volume.

% \section{Modelos de Deep Learning}
\subsubsection{GCN}
