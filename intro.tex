\chapter{Introdução}

As redes sociais se tornaram um dos principais meios de comunicação nas últimas
duas décadas.
Esse crescimento, em parte, se justifica pela massificação do acesso a internet
incluindo dispositivos móveis como \textit{smartphones} e \textit{tablets}.
Também alavancado pelos avanços computacionais e pelo desenvolvimento acelerado
de novas técnicas e algoritmos, o aprendizado de máquina, em especial o
processamento de linguagem natural, tem essas redes como importante objeto de
estudo.

Desde a chamada Revolução Digital observamos um progressivos barateamento e
facilidade de uso de dispositivos eletrônicos.
A medida que essas tecnologias passaram a ser acessíveis não apenas para as
corporações mas também para os indivíduos, houve um crescente processo de
digitalização de diversos aspectos de nossas vidas.
Com a comunicação não foi diferente, o email, por exemplo, desde os anos 70
substitui operações que até então eram apenas possíveis de forma analógica, como
pelo uso de cartas.
Nesse contexto, as redes sociais, ou mídias sociais, abordam aspectos diferente
da comunicação, uma comunicação mais dinâmica e informal.

Apesar de já estar presente desde os anos 90, é com a virada do milênio e que as
primeiras grandes mídias sociais online aparecem, como \textit{LinkedIn},
\textit{MySpace} e \textit{Orkut}.
Desde então há um crescimento de ano-a-ano na quantidade de seus usuários.
Atualmente estima-se que 3,5 bilhões de pessoas, ou 45\% da população mundial
utilize pelo menos uma rede social.
Este número torna-se ainda mais interessante quando considerado que 4,4 bilhões
de pessoas tem acesso a internet, portanto, quase 80\% dos usuários de internet
estão em alguma das mídias sociais.
No Brasil esses números se acentuam ainda mais, no qual 70\% da população tem
acesso a internet e 66\% utilizam as redes sociais.
% dados do we are social

Além da alta penetração na sociedade, devido a disponibilidade proporcionadas
pelos dispositivos móveis, consumimos boa parte de nosso tempo nessas redes.
No mundo, gastamos em média 2 horas e 16 minutos por dia e novamente esse número
é ainda superior no Brasil em que a média é de 3 horas e 34 minutos, sendo o
segundo país no mundo a usar por mais tempo as redes no mundo, ficando atrás
apenas das Filipinas.

... lazer, trabalho
.. com isso torna-se importante a análise dessas informações

\section{Motivação}
\section{Objetivo}
\section{Contribuições}
\section{Organização do Texto}

% ======================

Segundo a norma de formata{\c c}\~ao de teses e disserta{\c c}\~oes do
Instituto Alberto Luiz Coimbra de P\'os-gradua{\c c}\~ao e Pesquisa de
Engenharia (COPPE), toda abreviatura deve ser definida antes de
utilizada.\abbrev{COPPE}{Instituto Alberto Luiz Coimbra de P\'os-gradua{\c
c}\~ao e Pesquisa de Engenharia}

Do mesmo modo, \'e imprescind\'ivel definir os s\'imbolos, tal como o
conjunto dos n\'umeros reais $\mathbb{R}$ e o conjunto vazio $\emptyset$.
\symbl{$\mathbb{R}$}{Conjunto dos n\'umeros reais}
\symbl{$\emptyset$}{Conjunto vazio}
