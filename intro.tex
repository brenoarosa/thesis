\chapter{Introdução}
\label{chapter:intro}

Nas últimas duas décadas, as redes sociais se tornaram um dos principais meios de
comunicação.
Esse crescimento, em parte, se justifica pela massificação do acesso a internet
incluindo dispositivos móveis como \textit{smartphones} e \textit{tablets}.
Também alavancado pelos avanços computacionais e pelo desenvolvimento acelerado
de novas técnicas e algoritmos, o aprendizado de máquina, em especial o
processamento de linguagem natural, tem essas redes como importante objeto de
estudo.

Desde a chamada Revolução Digital, observamos um progressivo barateamento e
facilitação do uso de dispositivos eletrônicos.
À medida que essas tecnologias passaram a ser acessíveis, não apenas para as
corporações, mas também para os indivíduos, houve um crescente processo de
digitalização de diversos aspectos de nossas vidas.
Com a comunicação não foi diferente.
O email, por exemplo, substitui desde os anos 70 operações que até então eram
apenas possíveis de forma analógica, como pelo uso de cartas.
Nesse contexto, as redes sociais, ou mídias sociais, abordam aspectos diferentes
da comunicação, mais dinâmica e informal.

Apesar de já existirem desde os anos 90, é com a virada do milênio que as
primeiras grandes mídias sociais online aparecem, como \textit{LinkedIn},
\textit{MySpace} e \textit{Orkut}.
Desde então há um aumento anual da quantidade de seus usuários.
Atualmente, estima-se que 3,5 bilhões de pessoas, ou 45\% da população mundial
utilize pelo menos uma rede social.
Este número torna-se ainda mais interessante quando considerado que 4,4 bilhões
de pessoas têm acesso à internet.
Portanto, quase 80\% dos internautas estão em alguma das mídias sociais.
No Brasil esses números se acentuam ainda mais; 70\% da população tem
acesso à internet e 66\% utiliza as redes sociais~\cite{social19}.

Além da alta penetração na sociedade, devido à disponibilidade proporcionada
pelos dispositivos móveis, os usuários consomem boa parte de seu tempo nessas
redes.
No mundo, gasta-se em média 2 horas e 16 minutos por dia.
Novamente esse número é ainda superior no Brasil, onde a média é de 3 horas e
34 minutos, sendo o segundo país no mundo a usar por mais tempo as redes, ficando
apenas atrás das Filipinas.

Essa forte presença fez com que as mídias sociais não impactassem apenas as
comunicações.
Hoje em dia esses meios também são comumente utilizados para busca de
relacionamentos, compartilhamento de notícias, divulgação de serviços,
atendimento ao público, entre outros.
As informações que trafegam nas redes exercem grande influência na formação de
opinião das pessoas, seja ela em relação a um produto, a um evento ou até mesmo
temas políticos, como pôde-se observar nas eleições pelo mundo nos últimos anos.

Portanto, a análise dessas informações, presentes nas redes, é importante para
as mais diversos aplicações.
Contudo, essa grande quantidade de usuários também se reflete no número de dados
provindos das mídias sociais.
Dentre as estatísticas de uso do ano de 2018 fornecidas pelas próprias redes
sociais, tem-se que, diariamente, 300 milhões de fotos são publicadas no
\textit{Facebook}, 5 bilhões de vídeos são vistos no \textit{YouTube}, 43
bilhões de mensagens são enviadas no \textit{WhatsApp} e 100 milhões de usuários
interagem pelo \textit{Twitter}.
% dados de https://dustinstout.com/social-media-statistics/

O massivo volume de dados inviabiliza que essas análises sejam feitas
manualmente, tornando-se necessário o desenvolvimento de ferramentas capazes de
automatizar esse processo.
Entram aí as técnicas desenvolvidas pelo campo do Processamento de Linguagem
Natural, do inglês \textit{Natural Language Processing} (NLP).
\abbrev{NLP}{Natural Language Processing}
Foi a partir do anos 50 que esse termo passou a aparecer como um ramo da
Inteligência Artificial.
Devido à sua complexidade, a linguagem natural acabou inclusive se tornando um
critério de inteligência, como proposto no teste de Turing~\cite{turing50}.

O NLP é uma ampla área de pesquisa, abrangendo diferentes estágios da língua,
desde os níveis de abstração mais baixos, como o estudo da fonologia e da
sintaxe, quanto os níveis de abstração superior, que lidam com a semântica de
determinado conteúdo.
Neste ramo, busca-se desenvolver métodos capazes de auxiliar e/ou automatizar
tarefas como: o reconhecimento de fala, a análise sintática de frases, a
extração de entidades, a segmentação por tópicos, entre outros.

Esse conjunto de ferramentas são essenciais para tratar o volume de textos
gerado pelas redes sociais.  Com o avanço de técnicas de \textit{Deep Learning},
nos últimos anos pôde-se observar um avanço significativo de desempenho, que
permitiu até que tarefas de grande dificuldade, como a automatização de traduções
e de aplicações de atendimento ao cliente por conversação, sejam possíveis.

Entretanto, apesar do grande potencial dessas técnicas, as mídias sociais
apresentam características que diferenciam seu conteúdo dos tipos textuais que
tradicionalmente se analisam em NLP, como artigos e textos jornalísticos.
Isso dificulta o processamento dessa informação.
Em geral, as redes sociais se apresentam como meio de conversação ágil, logo,
as mensagem que circulam por elas costumam ser extremamente curtas, com amplo
uso de abreviações.
Por seu caráter informal, observa-se uma alta taxa de erros gramaticais e
uma grande utilização de \textit{emoticons}~\footnote{Sequência de caractéres ou
pequena imagem que transmite um estado emotivo}.
Somado a isso, o fato de serem meios de comunicação globais também ressalta a
presença de estrangeirismos~\footnote{Uso de palavras ou expressões estrangeiras}.
Ademais a dinamicidade das redes sociais faz com que a evolução de sentido das
palavras seja acelerada.
Esses elementos trazem a necessidade de adaptação ou desenvolvimento de novas
técnicas para se reproduzir o sucesso obtido pelas técnicas de processamento de
texto em documentos com escrita mais formal e estruturada.

Porém, o principal fator que distingue as informações de redes de outros meios é
a forte interligação entre diferentes tipos de mídia, como textos, imagens,
áudios, fotos e vídeos.
Além de metadados importantes, como localização, data e horário, uma propriedade
importante das mídias sociais são os atributos referentes às redes de usuário.
Exemplos desses dados são o número de amizades de um usuários da rede, e o
número de re-compartilhamentos de uma mensagem.
Logo, apesar da capacidade das ferramentas de NLP, existe um conjunto de
informações que essas técnicas desconsideram, abrindo espaço para que as
abordagens sejam multimodais, ou seja, que explorem diversas destas
propriedades.

De certa maneira, as interações entre usuários são o cerne das mídias sociais.
Logo, técnicas que se dispõem a analisar esse tipo de informação também são de
grande relevância.
Nesse quesito, o campo das Redes Complexas, ou Ciência de Redes, é responsável
por estudar os algoritmos e comportamentos observados em grafos que representam
sistemas reais, como no caso das redes sociais.
Assim como o aprendizado de máquina, essa esfera do conhecimento também
apresenta um grande crescimento nos últimos anos, fornecendo um novo leque de
tecnologias, de forma a descobrir-se aplicabilidades até então inexploradas.
Dentre suas aplicações, que possuem importância para o estudos das mídias
sociais, podemos ressaltar, por exemplo, a detecção de comunidades, a
identificação de principais influenciadores, e a modelagem de propagação de
informação.

Finalmente, estes métodos são poderasas ferramentas de análises de redes sociais,
principalmente quando aplicados em complemento à informação textual.
A complementariedade entre esses métodos permite, por exemplo, distinguir
conotações de uma mesma mensagem quando escritas por usuários de comunidades de
ideias opostas.
Por causa desses casos complexos, nos quais a análise de um elemento único da
mensagem não é suficiente para sua classificação é que estudos que se adequam a
multimodalidade das redes sociais se fazem necessários.

\section{Motivação}

Dados são considerados um dos bens mais valiosos da atualidade, de forma mesmo a
serem chamados de ``o novo petróleo''.
Isso porque, assim como o óleo, os dados são preciosos e precisam ser refinados
para terem utilidade.
Um dos aspectos dessa transformação pode ser observado na mudança cultural de
organizações e empresas que passam a tomar decisões baseadas em dados e métricas
coletadas.

A busca por informação de qualidade sobre um serviço ou produto sempre foi
importante para consumidores.
Quando não havia as tecnologias usadas atualmente essas pesquisas eram feitas
majoritariamente no boca-a-boca ou a partir de revistas especializadas.
Com a criação da internet e das redes sociais estas passaram também a exercer
essa função, com o beneficio de se encontrar opiniões de forma espontânea e em
grande quantidade.
As mídias sociais se tornaram um dos principais meios de compartilhamento dessa
informação.
As empresas, por sua vez, têm a oportunidade de utilizar as opiniões que
trafegam nas redes para identificar falhas em suas mercadorias, melhoras sua
segmentação, planejar novos produtos, entre outras atividades.
Com o fácil acesso a coleta desses dados, as ferramentas capazes de extrair o
sentimento dessas mensagens tornam-se fundamentais para viabilizar esse
procedimento na escala em que ocorrem.

Apesar das dificuldades inerentes a classificação de mensagens de redes sociais
técnicas de aprendizado de máquina, sobretudo \textit{Deep Learning}, e de Redes
Complexas apresentam êxito em várias tarefas realizadas sobre elas.
Entretanto, o sucesso desses modelos, em geral, depende da quantidade de dados
anotados disponíveis para treinamento.
Como o processo de anotação é manual esse passa a ser o gargalo da construção de
classificadores de sentimento.

Esse empecilho se torna ainda mais notável quando consideramos aplicações que
requerem a análise de múltiplas línguas ou que tenham foco em um tema
específico, necessitando criação de bases de dados próprias para cada caso de
uso.
Esses fatos motivam a elaboração de métodos que sejam independentes de bases de
treinamento.

\section{Objetivo}

Esse projeto visa desenvolver um método capaz de formar classificadores de
análise de sentimento sem a necessidade de bases de dados de treinamento.
Essas análises serão feitas sobre dados de mídia sociais e serão explorados
atributos tanto textuais quanto das redes de usuários.
A principal meta desse trabalho é viabilizar o emprego de modelos complexos,
que se beneficiam do grande volume de dados, sem o custo proveniente da anotação
de bases de treinamento.

Para análise das mensagens será avaliado o impacto da utilização classificadores
de \textit{Deep Learning} em comparação a métodos lineares tradicionalmente
aplicados em NLP.
Diferentes estratégias de representação de palavras e arquiteturas de
classificadores redes de aprendizado profundo serão testados de acordo com o
estado da arte em processamento de texto.
O processo será feito de maneira semi-supervisionada com supervisão distante
para anotação automática dos dados de treinamento.

Técnicas de Redes Complexas serão aplicadas para caracterização de autores das
mensagens.
Serão comparados modelos de representação de nós em grafos, támbem baseados em
aprendizado de máquina e de treinamento não supervisionado, cujo custo
computacional permitam operar em volumes de dados compatíveis com os de redes
sociais.
Neste caso, além de avaliar os modelos entre si, será analisado a informação
adicional provinda do usuários quando aplicada em conjunto com o classificador
textual decorre em algum benefício de performance do sistema.

Concluindo, há um amplo conjunto de estudo aplicando de processamento de
linguagem natural em redes sociais.
Apesar de menor, a ciência de redes também têm um grande reportório de pesquisa
sobre esse meio de comunicação.
Este trabalho visa preencher a lacuna de sistemas que não necessitam de
investimento em anotação de dados e que abrangem a multimodalidade da
informação característica desse tipo de dado.

%\section{Contribuições}
\section{Organização do Texto}

Esse documento é organizado da seguinte maneira:

\begin{itemize}
    \item O Capítulo~\ref{chapter:sentiment} apresenta o problema da análise
        de sentimento aplicada em mídias sociais e seus desafios.
        Esse Capítulo contém uma breve revisão bibliográfica de classificadores
        de análise de sentimento e de técnicas de redes aplicadas a essa tarefa.
    \item No Capítulo~\ref{chapter:nlp} as técnicas de processamento de
        linguagem natural são apresentadas.
        São caracterizadas as diferentes formas de representação numérica das
        mensagens, e sobre quais premissas se baseiam.
        Também são descritos os classificadores, tanto os métodos lineares
        tradicionalmente aplicados a textos quanto os de \textit{Deep Learning}.
    \item As ferramentas de ciência de redes são apresentadas no
        Capítulo~\ref{chapter:networks}.
        Nesse Capítulo são mostrados as diferentes técnicas de representação de
        vértices de uma rede, que serão aplicadas na modelagem de usuários de
        mídias sociais.
    \item O Capítulo~\ref{chapter:method} descreve o método proposto para
        desenvolvimento dos classificadores.
        São apresentados as etapas de formação de bases de dados, de anotação
        automática da mesma e de classificação.
    \item Os resultados dos obtidos pelos experimentos propostos são mostrados
        no Capítulo~\ref{chapter:results}.
    \item Por fim, o Capítulo~\ref{chapter:conclusion} avalia os resultados
        obtidos, apresenta as conclusões e enumera possíveis desdobramentos do
        trabalho realizado.
\end{itemize}
