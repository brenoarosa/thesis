\chapter{Introdução}

Nas últimas duas décadas as redes sociais se tornaram um dos principais meios de
comunicação.
Esse crescimento, em parte, se justifica pela massificação do acesso a internet
incluindo dispositivos móveis como \textit{smartphones} e \textit{tablets}.
Também alavancado pelos avanços computacionais e pelo desenvolvimento acelerado
de novas técnicas e algoritmos, o aprendizado de máquina, em especial o
processamento de linguagem natural, tem essas redes como importante objeto de
estudo.

Desde a chamada Revolução Digital observamos um progressivos barateamento e
facilitação do uso de dispositivos eletrônicos.
A medida que essas tecnologias passaram a ser acessíveis não apenas para as
corporações mas também para os indivíduos, houve um crescente processo de
digitalização de diversos aspectos de nossas vidas.
Com a comunicação não foi diferente.
O email, por exemplo, desde os anos 70 substitui operações que até então eram
apenas possíveis de forma analógica, como pelo uso de cartas.
Nesse contexto, as redes sociais, ou mídias sociais, abordam aspectos diferente
da comunicação, uma comunicação mais dinâmica e informal.

Apesar de já existirem desde os anos 90, é com a virada do milênio que as
primeiras grandes mídias sociais online aparecem, como \textit{LinkedIn},
\textit{MySpace} e \textit{Orkut}.
Desde então há um aumento de ano-a-ano da quantidade de seus usuários.
Atualmente estima-se que 3,5 bilhões de pessoas, ou 45\% da população mundial
utilize pelo menos uma rede social.
Este número torna-se ainda mais interessante quando considerado que 4,4 bilhões
de pessoas tem acesso a internet, portanto, quase 80\% dos usuários de internet
estão em alguma das mídias sociais.
No Brasil esses números se acentuam ainda mais, no qual 70\% da população tem
acesso a internet e 66\% utilizam as redes sociais~\cite{social19}.

Além da alta penetração na sociedade, devido a disponibilidade proporcionadas
pelos dispositivos móveis, consumimos boa parte de nosso tempo nessas redes.
No mundo, gastamos em média 2 horas e 16 minutos por dia e novamente esse número
é ainda superior no Brasil em que a média é de 3 horas e 34 minutos, sendo o
segundo país no mundo a usar por mais tempo as redes no mundo, ficando atrás
apenas das Filipinas.

Essa forte presença fez com que as mídias sociais não impactassem apenas as
comunicações.
Atualmente esses meios também são comumente utilizados para busca de
relacionamentos, compartilhamento de notícias, divulgação de serviços,
atendimento ao público, entre outros.
As informações que trafegam nas redes exercem grande influência na formação de
opinião das pessoas, seja ela em relação a um produto, um evento e até mesmo em
temas políticos como observamos nas eleições pelo mundo nos últimos anos.

Portanto, a análise dessas informações que presentes nas redes é importante para
as mais diversos aplicações.
Contudo, essa grande quantidade de usuários também se reflete no número de dados
provindos das mídias sociais.
Dentre as estatísticas de uso do ano de 2018 fornecidas pelas próprias redes
sociais temos que diariamente 300 milhões de fotos são publicadas no
\textit{Facebook}, 5 bilhões de vídeos são vistos no \textit{YouTube}, 43
bilhões de mensagens são enviadas no \textit{WhatsApp} e 100 milhões de usuários
interagem pelo \textit{Twitter}.
% dados de https://dustinstout.com/social-media-statistics/

O massivo volume de dados inviabiliza que essas análises sejam feitas
manualmente, se tornando necessário o desenvolvimento de ferramentas capazes de
automatizar esse processo.
Entram aí as técnicas desenvolvidas pelo campo do Processamento de Linguagem
Natural, do inglês \textit{Natural Language Processing} (NLP).
\abbrev{NLP}{Natural Language Processing}
Foi a partir do anos 50 que esse termo passou a aparecer como um ramo da
Inteligência Artificial.
Devido a complexidade da linguagem natural acabou se tornando um critério de
inteligência, como proposto no teste de Turing~\cite{turing50}.

O Processamento de Linguagem Natural é uma ampla área de pesquisa, abrangendo o
estudo de diferentes estágios da língua, desde os níveis de abstração mais
baixos como o estudo da fonologia e da sintaxe quanto os maiores que lidam com a
semântica de determinado conteúdo.
Neste ramo busca-se desenvolver métodos capazes de auxiliar e/ou automatizar
tarefas como: o reconhecimento de fala, análise sintática de frases, extração de
entidades, segmentação por tópicos, entre outras.

Esse conjunto de ferramentas desenvolvidas pelo NLP que são essenciais para
tratar o volume de textos gerado pelas redes sociais.
Em especial, nos últimos anos como o avanço de técnicas de \textit{Deep
Learning} observamos um avanço significativo de desempenho, que permitiu até que
tarefas de grande dificuldade como a automatização de traduções e aplicações de
atendimento ao cliente por conversação sejam possíveis.

Entretanto, apesar do grande potencial dessas técnicas, as mídias sociais
apresentam características que diferenciam seu conteúdo dos tipos textuais que
tradicionalmente se analisam em NLP, como artigos e textos jornalísticos e que
dificultam o processamento dessa informação.
Outras características, apesar de presentes em outras formas de texto aparecem
aqui acentuadas.
Em geral, as redes sociais se apresentam como meio de conversação ágil, logo,
as mensagem que circulam pela mesma costumam ser extremamente curtas, com amplo
uso de abreviações.
Por seu caráter informal, observamos alta taxa de erros gramaticais e utilização
de \textit{emoticons}.
O fato de serem meios de comunicação global também ressalta a presença de
estrangeirismos.
A dinamicidade das redes sociais faz com que a evolução de sentido das palavras
seja acelerada.
A Figura ... mostra um exemplo de mensagem que contém alguns desses pontos.
% descrever o que pode ser ressaltado da figura na legenda

Porém, o principal fator que distingue as informações de redes de outros meios é
a forte interligação entre diferentes tipos de mídia como textos, imagens,
áudios, fotos e vídeos.
Além de metadados importantes como localização, data e horário, uma propriedade
importante de redes sociais são os atributos referentes as redes de usuário.
Exemplos desses dados são o número de amizades de um usuários da rede, e o
número de re-compartilhamentos de uma mensagem.

De certa maneira as interações entre usuários são o cerne das mídias sociais.
Logo, técnicas que se dispõem a analisar esse tipo de informação também são de
grande relevância.
Nesse quesito, o campo das Redes Complexas permite uma outra abordagem sobre
esse tipo de conteúdo, sendo complementar a análise textual.
Ele é responsável por estudar os algoritmos e comportamentos observados em
grafos que representam sistemas reais, como no caso das redes sociais.
Assim como o aprendizado de máquina, essa esfera do conhecimento também
apresenta um grande crescimento nos últimos anos, fornecendo um novo leque de
tecnologias e descobrindo-se aplicabilidades até então inexploradas.
Dentre suas aplicações que possuem importância para o estudos das mídias sociais
podemos ressaltar, por exemplo, a detecção de comunidades, identificação de
principais influenciadores, modelagem de propagação de informação, entre outras.

\section{Motivação}
% dado a grande utilidade falar de produtos, impactos no mercado e etc
\section{Objetivo}
\section{Organização do Texto}
