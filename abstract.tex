\begin{foreignabstract}

During the last years, social media became one of the main communication channels, growing its influence over its users.
Accordingly, the messages that run on social media started to be relevant for a wide range of applications, such as product reviews.
Consequently, initiating a variety of analyses on this modality of text, one of them being the sentiment analysis which has clear direct applications.
In this manner, techniques of natural language processing and complex networks can assist in these analyses.
However, the performance of these tools usually depends on the volume of manually labeled training data, therefore adding costs to its utilization.
This work applies a method of automatic data labeling to generate training datasets and avoid this barrier.
Multiple text classification models are evaluated, both linear and non-linear, such as Deep Learning.
Additionally, we assess complex network techniques on the task of representing users.
Therefore, multiple aspects of the information available on these social networks are considered by the models.

\end{foreignabstract}

