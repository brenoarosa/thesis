\chapter{Resultados e Discussões}
\label{chapter:results}

% BASE
% termos e temas coletados
% tabela de escolha de emoticons
% volume da base taggeada automaticamente
% relacao usuarios (retweets)
% status dos grafos
% volume da base de teste

A etapa de coleta de dados captou aproximadamente 9,2 milhões de tweets únicos
entre 2018 e 2019.
Parte desses tweets foi coletada da amostragem do conjunto total de mensagens
publicadas e outra parte da amostrada por dentre os tópicos: \textit{Libertadores,
ENEM, Amazônia} e \textit{Rock in Rio}.
Nessa base há aproximadamente 45 milhões de retweets, fornecendo a conexão entre
pessoas que será utilizada para montar a rede de usuários.

Para anotação automática foram identificados os emoticons mais frequentes da
base de dados e classificados manualmente entre positivos e negativos,
desconsiderando os neutros.
A Tabela~\ref{tab:emoticons} mostra as classes dos emoticons selecionados.
Com esse conjunto de emoticons 580 mil tweets foram anotados por supervisão
distante, dentre os quais 130 mil marcados como negativos e 450 mil marcados como
positivos.

\begin{table}[h]
    \begin{center}
        \begin{tabular}{| l | c |}
        \hline
        \textbf{Classe} & \textbf{Emoticons} \\ \hline
        Positiva &
            \includegraphics[height=1em]{images/emojis/2764}
            \includegraphics[height=1em]{images/emojis/1F602}
            %\includegraphics[height=1em]{images/emojis/1F5A4}
            %\includegraphics[height=1em]{images/emojis/1F923}
            \includegraphics[height=1em]{images/emojis/1F60D}
            \includegraphics[height=1em]{images/emojis/2665}
            %\includegraphics[height=1em]{images/emojis/1F970}
            \includegraphics[height=1em]{images/emojis/1F605}
            \includegraphics[height=1em]{images/emojis/1F601}
            %\includegraphics[height=1em]{images/emojis/1F92D}
            \includegraphics[height=1em]{images/emojis/1F618}
            \includegraphics[height=1em]{images/emojis/1F609}
            \includegraphics[height=1em]{images/emojis/1F496}
            \includegraphics[height=1em]{images/emojis/1F495}
            \includegraphics[height=1em]{images/emojis/1F606}
            %\includegraphics[height=1em]{images/emojis/1F973}
            \includegraphics[height=1em]{images/emojis/1F499}
            \includegraphics[height=1em]{images/emojis/1F389}
            \includegraphics[height=1em]{images/emojis/1F61D}
            \includegraphics[height=1em]{images/emojis/1F49A}
            \includegraphics[height=1em]{images/emojis/1F49C}
            %\includegraphics[height=1em]{images/emojis/2763}
            \includegraphics[height=1em]{images/emojis/1F60A}
            \includegraphics[height=1em]{images/emojis/1F60B}
            %\includegraphics[height=1em]{images/emojis/1F917}
        \\ \hline
        Negativa &
            \includegraphics[height=1em]{images/emojis/1F62D}
            \includegraphics[height=1em]{images/emojis/1F645}
            %\includegraphics[height=1em]{images/emojis/1F926}
            \includegraphics[height=1em]{images/emojis/1F621}
            \includegraphics[height=1em]{images/emojis/1F614}
            %\includegraphics[height=1em]{images/emojis/1F92C}
            %\includegraphics[height=1em]{images/emojis/1F92E}
            \includegraphics[height=1em]{images/emojis/1F629}
            \includegraphics[height=1em]{images/emojis/1F622}
            \includegraphics[height=1em]{images/emojis/1F620}
            \includegraphics[height=1em]{images/emojis/1F612}
            \includegraphics[height=1em]{images/emojis/1F624}
            \includegraphics[height=1em]{images/emojis/1F494}
            \includegraphics[height=1em]{images/emojis/1F62A}
            \includegraphics[height=1em]{images/emojis/1F625}
            \includegraphics[height=1em]{images/emojis/1F62B}
            \includegraphics[height=1em]{images/emojis/1F630}
        \\ \hline
        \end{tabular}
        \caption{Emoticons selecionados para aplicação de supervisão distante.}
        \label{tab:emoticons}
    \end{center}
\end{table}

A rede de usuários foi formada por retweets entre usuários.
Os 45 milhões de retweets formam uma rede de 28 milhões de arestas únicas
conectando 5,5 milhões de vértices, representando uma densidade de
$9,4\mathrm{e}{-7}$.
O coeficiente de clusterização global é de $9,6\mathrm{e}{-4}$, portanto,
havendo um vizinho em comum o nó tem 1000 vezes mais chance de estar conectado a
outro nó do que quando ambos não compartilham conexões.
Aproximadamente 4,5\% dos nós pertencem a componente fortemente conexa, ou seja,
sub-grafo direcional em que existe um caminho entre todos os vértices.
Enquanto 97\% dos nós pertencem a componente fracamente conexa, que leva em
consideração o sub-grafo não direcional.
Observa-se que a rede inteira é formada praticamente por uma única componente
gigante enquanto os outros 3\% dos usuários estão distribuídos em um grande
volume de pequenas componentes.
A figura~\ref{fig:graph_ccdf} mostra as distribuições de grau dos vértices.
Analisa-se que tanto a distribuição de grau de entrada quanto a de saída tem
caudas longas apesar da curva da distribuição de de grau de entrada ser
significativamente mais extensa.
Ressalta-se também que mais que 80\% dos nós tem grau de entrada 0, ou seja,
são usuários que não receberam nenhum retweet.

\begin{figure}[h]
\begin{center} {
    \begin{center}
    \includegraphics[scale=0.65]{images/graph_ccdf.png}
    \caption{Gráfico da \textit{CCDF (complementary cumulative distribution function)}
             dos graus de entrada e saída da rede.
             O eixo horizontal representa o grau $k$ enquanto o eixo vertical
             demonstra a probabilidade de um vertice ter grau pelo menos $k$.}
    \label{fig:graph_ccdf}
    \end{center}
}
\end{center}
\end{figure}

% todo: grafico de nos de distancia


% CLASSIFICADOR
