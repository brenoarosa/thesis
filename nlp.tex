\chapter{Processamento de Linguagem Natural}
\label{chapter:nlp}

Neste capitulo serão apresentadas as técnicas de Processamento de Linguagem
Natural que compõe um classificador, como o de análise de sentimento.
De modo geral, esse processo é composto de 3 etapas, como demostrado no diagrama
(TODO).
As seções a seguir descrevem cada uma destas fases.
% TODO: diagrama de etapas de um classificador por NLP

\section{Pré-Processamentos}

A primeira etapa aplicada para elaboração de modelos de NLP é o
pré-processamento.
Esta fase consiste na limpeza e preparação dos dados, visando melhorar a
performance do classificador seja retirando ruidos dos textos, reduzindo o
tamanho do vocabulário ou formatando o texto de maneira a facilitar a modelagem.
O volume do vocabulário considerado é costuma ser limitado seja pelos recursos
computacionais quanto pelo requisito mínimo de estatistica das palavras na base
de dados.
Portanto, técnicas de pré-processamento que reduzam o tamanho total do
vocabulário tem um importante papel na garantia de bom funcionamento dos
classificadores.

Como a maior parte dos modelos de NLP trabalham a nível de palavra é necessário
separar separar o documento em frases, com algoritmos como o Punkt~\cite{kiss06},
e posteriormente, em palavras.
Esse processo chamado \textit{tokenização} precisa ser robustos a abreviações,
números e características do idioma ao qual será aplicado, como contração de
palavras.

Algoritmos de correção ortográfica~\cite{damerau64}\cite{navarro01} podem ser
eficientes para aprimorar a qualidade dos textos, principalmente se tratando de
meios de comunicação dinâmicos como as redes sociais.

Técnicas de stemização consistem na extração do o radical das palavras,
como o obtido pelo algoritmo de Porter~\cite{porter80}, um exemplo é dado com a
palavra "montanha" que possuí radical "mont", o mesmo obtido pela palavra
"monte".
Por outro lado, o processos de lematização tem finalidade parecida, porém
transforma a palavra em sua forma base, forma como elas aparecem no dicionario,
podendo então difenciar palavras com o mesmo radical, como "banco" e "bancários".
Ambas as técnicas visam tornar as etapas posteriores menos sensíveis a flexões
gramáticais, além de colaborar na redução do vocabulario.

Uma das principais etapas
% stopwords

% falar de coisas como Part of speech (pode ser usado como feature)? talvez falar em classificadores
% pode citar que é uma forma de remover stopwords

\section{Representações}

\subsection{One-Hot}
\subsection{Bag-of-Words}
\subsection{Word2Vec}
\subsection{Seq2Seq (BERT)}
\section{Classificadores}
\subsection{Baseados em Dicionário}
\label{sec:dictionary}
\subsection{Lineares}
\subsection{Não Lineares}
\subsection{Deep Learning}
% diferença principal de deep p tradicional é: contexto
% inspiração: http://dataskeptic.com/blog/episodes/2019/natural-language-processing

% falar de diferentes dificultades de NLP e citar trabalhos relacionados no cap 3
% falar de tecnicas (nlp): knowledge-based (lexicon), lineares, n lineares...
