\chapter{Análise de Sentimento de Redes Sociais}
\label{chapter:sentiment}

Análise de Sentimento é o campo de estudos que analisa a opinião, sentimento,
atitude e emoções de pessoas em relação a entidades.
Essas entidades podem ser pessoas, eventos, produtos, tópicos, entre outros.
Na literatura também se encontram os seguintes nomes relacionados a esse ramo:
\textit{mineiração de opinião}, \textit{extração de opinião},
\textit{mineiração de sentimento}, \textit{análise de subjetividade},
\textit{análise de emoção} e \textit{extração de críticas}~\cite{liu15}.
Por ser aplicado na maioria das vezes a textos escritos, este campo é filiado
ao processamento de linguagem natural, sendo uma de suas ramificações mais
ativas.

Segundo \citet{cambria16}, a Análise de Sentimento é dividida entre duas
principais tarefas, a extração de polaridade e a detecção de emoções.
Enquanto a primeira foca em discernir conteúdo positivo de negativo, a segunda é
responsavel por classificar em emoções como: felicidade, medo, raiva, tristeza,
entre outros.
A autora \citet{liu15}, por sua vez, ressalta que (divide em documento, frase, aspecto

% Levantamento das tarefas envolvidas em analise de sentimento
%
% - Opinions, Sentiment, and Emotion in Text
% - Many facets of Sentiment Analysis (A practical guide to sentiment analysis)
% - sentiment analysis and opinion mining (liu)

% falar de tecnicas (nlp): knowledge-based (lexicon), lineares, n lineares...
% falar de redes
