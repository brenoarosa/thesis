\chapter{Análise de Sentimento de Redes Sociais}
\label{chapter:sentiment}

Análise de Sentimento é o campo de estudos que analisa a opinião, sentimento,
atitude e emoções de pessoas em relação a entidades.
Essas entidades podem ser pessoas, eventos, produtos, tópicos, entre outros.
Na literatura também se encontram os seguintes nomes relacionados a esse ramo:
\textit{mineração de opinião}, \textit{extração de opinião},
\textit{mineração de sentimento}, \textit{análise de subjetividade},
\textit{análise de emoção} e \textit{extração de críticas}~\cite{liu15}.
Por ser aplicado na maioria das vezes a textos escritos, este campo é filiado
ao processamento de linguagem natural, sendo uma de suas ramificações mais
ativas.

Segundo \citet{cambria16}, a Análise de Sentimento é dividida entre duas
principais tarefas, a extração de polaridade e a detecção de emoções.
Enquanto a primeira foca em discernir conteúdo positivo de negativo, a segunda é
responsável por classificar em emoções como: felicidade, medo, raiva, tristeza,
entre outros.

A autora \citet{liu15}, por sua vez, ressalta que há diferentes níveis de
granularidade que essas tarefas podem ser executadas, a escolha do nível a ser
utilizado depende da finalidade pela qual se executa a classificação e será uma
das principais caraterísticas para definir o tipo de técnica empregada.
Pode se realizar a Análise de Sentimento a nível de documento, no qual um
documento, como uma avaliação de produto, são avaliados como um todo.
Nesses casos se assume implicitamente que um documento expressa uma opinião
sobre uma única entidade, como o produto em questão no exemplo citado, também
fica implícita que o documento expressa um sentimento único sobre a entidade.
Por serem textos mais extensos, logo com mais informação, a acertividade de
modelos nesses casos em geral é mais alta, portanto, até técnicas mais simples
como as baseadas em dicionario como será apresentado em (TODO: referenciar exemplo com lexicon
(subcapitulo seguinte 2.1)) podem apresentar resultados suficientemente bons.
Sendo assim, este foi o primeiro nível a ser explorado pela literatura.
% mesmo botando a tecnica em outro capitulo preciso pipocar exemplos de
% referencias

Visto que as limitações decorrentes de se classificar documentos por inteiro
reduzem o escopo de aplicações, pode-se recorrer ao nível seguinte de
granularidade, a classificação de sentimento de sentenças.
A principal diferença entre essa abordagem e a anterior é a quantidade de
informação disponível dado que uma sentença é composta, geralmente, por poucas
palavras.
Por outro lado, a assunção de sentimento único no conteúdo de uma frase é mais
coerente com a realidade do que se considerando o documento inteiro, sendo uma
aproximação boa o suficiente para uma nova gama de casos de uso.
Entretanto, \citet{liu15} ressalta que para o caso de sentenças é importante que
a classificação de polaridade leve em consideração o sentimento neutro.
Isso se torna relevante pois até mesmo dentro de documentos opinativos, como
avaliações de produto, há sentenças puramente objetivas, que não expressarão
polaridade sobre uma entidade.
Devido a quantidade limitada de informação, classificadores baseados em
dicionários e em técnicas de aprendizado de máquina por modelos lineares,
como será introduzido em (TODO: secao), apresentam piores indicadores na
execução da tarefa.
Nesse cenário os modelos de \textit{Deep Learning} começaram a sobressair.

% granularidade aspecto:
% falar que no exemplo de review de produto serve pra diferenciar pontos
% positivos de negativos

% Levantamento das tarefas envolvidas em analise de sentimento
%
% - Opinions, Sentiment, and Emotion in Text
% - Many facets of Sentiment Analysis (A practical guide to sentiment analysis)
% - sentiment analysis and opinion mining (liu)

% juntar 2 e 3?
% falar de tecnicas (nlp): knowledge-based (lexicon), lineares, n lineares...
% falar de redes
