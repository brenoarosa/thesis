\chapter{Análise de Sentimento de Redes Sociais}
\label{chapter:sentiment}

Análise de Sentimento é o campo de estudos que analisa a opinião, sentimento,
atitude e emoções de pessoas em relação a entidades.
Essas entidades podem ser pessoas, eventos, produtos, tópicos, entre outros.
Na literatura também se encontram os seguintes nomes relacionados a esse ramo:
\textit{mineração de opinião}, \textit{extração de opinião},
\textit{mineração de sentimento}, \textit{análise de subjetividade},
\textit{análise de emoção} e \textit{extração de críticas}~\cite{liu15}.
Por ser aplicado na maioria das vezes a textos escritos, este campo é filiado
ao processamento de linguagem natural, sendo uma de suas ramificações mais
ativas.

Segundo \citet{cambria16}, a Análise de Sentimento é dividida entre duas
principais tarefas, a extração de polaridade e a detecção de emoções.
Enquanto a primeira foca em discernir conteúdo positivo de negativo, a segunda é
responsável por classificar em emoções como: felicidade, medo, raiva, tristeza,
entre outros.

A autora \citet{liu15}, por sua vez, ressalta que há diferentes níveis de
granularidade que essas para execução desta tarefa, a escolha do nível a ser
utilizado depende da finalidade pela qual se aplicará a classificação e será uma
das principais caraterísticas para definir o tipo de técnica empregada.
A Análise de Sentimento pode ser realizada a nível de documento, no qual um
documento, como uma avaliação de produto, é avaliado como um todo.
Nesses casos se assume implicitamente que um documento expressa uma opinião
sobre uma única entidade, como o produto em questão no exemplo citado, também
fica implícita que o documento expressa um sentimento único sobre a entidade.
Por serem textos mais extensos, logo com mais informação, a acertividade de
modelos nesses casos em geral é mais alta, portanto, até técnicas mais simples
como as baseadas em dicionario como será apresentado em (TODO: referenciar exemplo com lexicon
% TODO: link secao
(subcapitulo seguinte 2.1)) podem apresentar resultados suficientemente bons.
\citet{taboada11} exemplifica a extração de opinião de documentos a partir de
técnicas de dicionário aplicadas em diferentes bases de dados de avaliações de
produtos.
Por sua vez, \citet{das07} cacula predição de valor de ações a partir de
análise de sentimento das mensagens presentes em um fórum online de investidores.

Visto que as limitações decorrentes de se classificar documentos por inteiro
reduzem o escopo de aplicações, pode-se recorrer ao nível seguinte de
granularidade, a classificação de sentimento de sentenças.
A principal diferença entre essa abordagem e a anterior é a quantidade de
informação disponível dado que uma sentença é composta, geralmente, por poucas
palavras.
Por outro lado, a premissa de sentimento único no conteúdo de uma frase é mais
coerente com a realidade comparando-se com a classificação documento como um
todo, sendo uma aproximação boa o suficiente para uma nova gama de casos de uso.
Entretanto, \citet{liu15} ressalta que para o caso de sentenças é importante que
a classificação de polaridade leve em consideração o sentimento neutro.
Isso se torna relevante pois até mesmo dentro de documentos opinativos, como
avaliações de produto, há sentenças puramente objetivas, que não expressarão
polaridade sobre uma entidade.
\citet{riloff05} apresentam o primeiro trabalho especificamente voltado para
classificação de subjetividade de documentos.
Devido a quantidade limitada de informação presentes em uma sentença,
classificadores baseados em dicionários e em técnicas de aprendizado de máquina
por modelos lineares, apresentam indicadores piores na execução da tarefa.
São nesses cenário que nos últimos anos os modelos não lineares começaram a
sobressair, como apresentam \citet{socher11} e \citet{socher13} que aplicam
diferentes técnicas baseadas em \textit{Deep Learning} para classificação de
sentenças.

Por fim, apresenta-se a análise de sentimento de características.
Uma entidade pode ser composta de diversos atributos.
Ao falar sobre um filme pode se avaliar diferentes aspectos dele, como
o roteiro, os atores, os personagens, etc.
Uma crítica a esse determinado filme é composta de sentimentos distintos para
cada um desses atributos, e o objetivo da análise de sentimento de
características é identificar a polaridade de uma mensagem em relação aos
atributos presentes.
Para realizar essa análise são necessários elementos novos, como o reconhecimento
da entidade citada e quais aspectos dela estão sendo avaliados.
Também podem ser relevantes a identificação do autor e do momento do documento
analisado.
\citet{nasukawa03} e \citet{snyder07} são exemplos de trabalhos focados em
mineiração de opinião focada em aspectos.

% puxar o link com o anterior, falar que tweet se assemelha ao segundo nivel,
% comecar exemplos de referencias de SA em twitter e outras redes
A Análise de Sentimento aplicada a redes sociais, em especial ao Twitter, se
assemelha a classificação de sentimento de sentenças.
Entretanto o perfil de mensagem que circulam as mídias sociais apresentam
peculiaridades quando comparadas a meios convencionais.
Por se tratar de um ambiente informal e de comunicação rápida, é comum encontrar
erros gramaticais e abreviações.
Similarmente os \textit{emoticons}, icones ilustrativos de expressões faciais,
também tem ampla adesão por serem métodos práticos de exprimir sentimentos em
poucos caractéres.
Por se tratarem de redes globais, é frequente o emprego de palavras ou
expressões de outras linguas em uma mesma mensagem.
Esses fatores são obstacúlos aos classificadores de linguagem natural,
dificultando a tarefa de extração de polaridade.

O Twitter é uma rede social baseada em interações por mensagens curtas.
Suas principais características são a brevidade e instantaniedade das mensagens,
também chamadas \textit{tweets}, que são limitadas atualmente em 280 caracteres
mas famosas pelo seu limite anterior de 140 caracteres.
Criada em 2016, a rede conta com 139 milhões de usuários ativos diarios segundo
seu...(TODO)  e está entre as redes sociais com maior número de usuários do mundo.
% https://s22.q4cdn.com/826641620/files/doc_financials/2019/q2/Q2-2019-Shareholder-Letter.pdf

O Twitter foi responsável pela criação e popularização das \textit{hashtags},
mecânismo que funciona como marcação de palavra-chave ou tópico.
Seu funcionamento se dá pela utilização do simbolo da tralha (\#) e pela ausência
de espaço e pontuação nos casos que são formados por mais que uma palavra.
As \textit{hashtags} functionam como um agregador de \textit{tweets} e o site
ainda apresenta uma lista das mais populares no momento.
O sucesso dessa funcionalidade fez com que a mesma fosse posteriormente aderida
por outras mídias sociais como o Facebook e Instagram se tornando um
atributo marcante mensagens de redes sociais.

% faltou metadados
Além das \textit{hashtags}, as mensagens no Twitter também possibilitam a
inserção de mídias como imagens, vídeos e audios.
Outro atributo comum em mídias sociais são as redes formadas pela conexão de
usuários.
Essas redes podem ser formadas por relações de amizade, seguidores e outras
interações entre essas pessoas ou entidades.
O recompartilhamento de mensagem, que se dá quando um usuário divulga em seu
perfil uma mensagem criada por outra pessoa é outro componente capaz de gerar
grafos de usuários, no Twitter o recompartilhamento também é chamado de
\textit{retweet}.

A mistura entre diferentes modalidades de comunicação: texto, imagem, video, etc
gera uma dificuldade adicional para análise das mensagems.
O texto presente em um \textit{tweet} que possuí uma imagem pode fazer
interlocução com a mesma.
Uma mensagem enviada por um usuário pode ter sentido oposto a mesma mensagem
quando comunicada por alguem que pertença a um grupo opositor.
Nesses casos a análise do conteúdo textual por sí só é incapaz de captar a
essência da mensagem.
Dá-se assim a necessidade de abordagens multimodais para qualquer tipo de
análise de redes sociais.
% rever intro pois isso ja foi falado

% conectar com redes (grafos), aproveitar o gancho de redes sociais
% valorizar grafo como ponto principal das redes sociais.

% Target-dependent Twitter Sentiment Classification
% Bom pra pegar referencia de twitter e grafos

% Levantamento das tarefas envolvidas em analise de sentimento
%
% - Opinions, Sentiment, and Emotion in Text
% - Many facets of Sentiment Analysis (A practical guide to sentiment analysis)
% - sentiment analysis and opinion mining (liu)

% falar de diferentes dificultades de NLP e citar trabalhos relacionados
% pode ser um sub capitulo, talvez no capitulo 3

% juntar 2 e 3?
% falar de tecnicas (nlp): knowledge-based (lexicon), lineares, n lineares...
% falar de redes
