\chapter{Análise de Sentimento de Redes Sociais}
\label{chapter:sentiment}

Análise de Sentimento é o campo de estudos que analisa a opinião, sentimento,
atitude e emoções de pessoas em relação a entidades.
Essas entidades podem ser pessoas, eventos, produtos, tópicos, entre outros.
Na literatura também se encontram os seguintes nomes relacionados a esse ramo:
\textit{mineiração de opinião}, \textit{extração de opinião},
\textit{mineiração de sentimento}, \textit{análise de subjetividade},
\textit{análise de emoção} e \textit{extração de críticas}~\cite{liu15}.
Por ser aplicado na maioria das vezes a textos escritos, este campo é filiado
ao processamento de linguagem natural, sendo uma de suas ramificações mais
ativas.

Segundo \citet{cambria16}, a Análise de Sentimento é dividida entre duas
principais tarefas, a extração de polaridade e a detecção de emoções.
Enquanto a primeira foca em discernir conteúdo positivo de negativo, a segunda é
responsavel por classificar em emoções como: felicidade, medo, raiva, tristeza,
entre outros.

A autora \citet{liu15}, por sua vez, ressalta que há diferentes níveis de
granularidade que essas tarefas podem ser executadas, a escolha do nível a ser
utilizado depende da finalidade pela qual se executa a classificação e será uma
das principais caracteristas para definir o tipo de técnica empregada.
Pode se realizar a Análise de Sentimento a nível de documento, no qual um
documento, como uma avaliação de produto, são avaliados como um todo.
Nesses casos se assume implicitamente que um documento expressa uma opinião
sobre uma única entidade, como o produto em questão no exemplo citado, também
fica implicita que o documento expressa um sentimento único sobre a entidade.
Por serem textos mais extensos, logo com mais informação, a acertividade de
modelos nesses casos em geral é mais alta, portanto, até técnicas mais simples
como as baseadas em dicionario (todo: referenciar exemplo com lexicon) podem
apresentar resultados suficientemente bons.
Sendo assim, este foi o primeiro nível a ser explorado pela literatura.

O nível seguinte de granularidade é a classificação de sentenças, esse caso é

% Levantamento das tarefas envolvidas em analise de sentimento
%
% - Opinions, Sentiment, and Emotion in Text
% - Many facets of Sentiment Analysis (A practical guide to sentiment analysis)
% - sentiment analysis and opinion mining (liu)

% falar de tecnicas (nlp): knowledge-based (lexicon), lineares, n lineares...
% falar de redes
