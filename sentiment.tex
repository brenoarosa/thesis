\chapter{Análise de Sentimento de Redes Sociais}
\label{chapter:sentiment}

A Análise de Sentimento é o campo de estudos que analisa opinião, sentimento,
atitude e emoções de pessoas em relação a entidades.
Essas entidades podem ser outras pessoas, eventos, produtos, tópicos etc.
Na literatura, também se encontram os seguintes nomes relacionados a esse ramo:
\textit{mineração de opinião}, \textit{extração de opinião},
\textit{mineração de sentimento}, \textit{análise de subjetividade},
\textit{análise de emoção} e \textit{extração de críticas}~\cite{liu15}.
Por ser aplicado majoritariamente a textos escritos, este campo é filiado
ao processamento de linguagem natural, sendo uma de suas ramificações mais
ativas.

Segundo \citet{cambria16}, a Análise de Sentimento é dividida entre duas
principais tarefas, a extração de polaridade e a detecção de emoções.
Enquanto a primeira se concentra em discernir conteúdo positivo de negativo,
a segunda é responsável por classificar o documento em emoções como felicidade,
medo, raiva e tristeza.

A autora \citet{liu15}, por sua vez, ressalta que há diferentes níveis de
granularidade de execução desta tarefa.
A escolha do nível a ser utilizado depende da finalidade pela qual se aplicará
a classificação e será uma das principais caraterísticas para definir o tipo
de técnica empregada.
A Análise de Sentimento pode ser realizada a nível de documento, no qual um
documento, como uma avaliação de produto, é analisado como um todo.
Nesses casos, assume-se implicitamente que um documento expressa uma opinião
sobre uma única entidade, como o produto em questão no exemplo citado.
Também fica implícito que o documento expressa um sentimento único sobre a
entidade.
Por serem textos mais extensos, logo com mais informação, a assertividade de
modelos nesses casos geralmente é mais alta.
Portanto, mesmo as técnicas mais simples como as baseadas em dicionário, como
serão apresentadas na seção~\ref{sec:dictionary}, podem obter resultados
suficientemente bons.
\citet{taboada11} exemplifica a extração de opinião de documentos a partir de
técnicas de dicionário aplicadas em diferentes a bases de dados de avaliações de
produtos.
Por sua vez, \citet{das07} realiza predições de valores de ações a partir de
análise de sentimento das mensagens de um fórum online de investidores.

Visto que as limitações decorrentes de se classificar documentos por inteiro
reduzem o escopo de aplicações possíveis, pode-se recorrer ao nível seguinte de
granularidade: a classificação de sentimento de sentenças.
A principal diferença entre essa abordagem e a anterior é a quantidade de
informação disponível, dado que uma sentença é composta, geralmente, por poucas
palavras.
Por outro lado, a premissa do sentimento único no conteúdo de uma frase é mais
coerente com a realidade quando comparada com a classificação do documento como um
todo, sendo uma aproximação suficientemente boa para uma nova gama de casos de uso.
Entretanto, \citet{liu15} ressalta que para o caso de sentenças é importante que
a classificação de polaridade leve em consideração o sentimento neutro.
Isso se torna relevante pois, até mesmo dentro de documentos opinativos, como
avaliações de produto, há sentenças puramente objetivas, que não expressam
polaridade sobre uma entidade.
Já \citet{riloff05} apresentam o primeiro trabalho especificamente voltado para
classificação de subjetividade de documentos.
Devido à quantidade limitada de informação presentes em uma sentença,
classificadores baseados tanto em dicionários, quanto em técnicas de
aprendizado de máquina por modelos lineares apresentam indicadores piores na
execução de tal tarefa.
É nesse cenário que nos últimos anos os modelos não lineares começaram a
sobressair, como apresentam \citet{socher11} e \citet{socher13}, que aplicam
diferentes técnicas baseadas em \textit{Deep Learning} para classificação de
sentenças.

Por fim, apresenta-se a análise de sentimento de características.
Uma entidade pode ser composta de diversos atributos.
Ao falar sobre um filme, pode-se avaliar aspectos dele como o roteiro, os
atores, os personagens etc.
Uma crítica a esse determinado filme é composta de sentimentos distintos para
cada um desses atributos e o objetivo da análise de sentimento de
características é identificar a polaridade de uma mensagem em relação a cada um
dos atributos presentes.
Para realizar essa análise, são necessários elementos novos, como o reconhecimento
de entidade e dos aspectos dos quais ela é composta.
Também podem ser relevantes a identificação do autor e do momento do documento
analisado.
\citet{nasukawa03} e \citet{snyder07} são exemplos de trabalhos que aplicam
mineração de opinião focada em aspectos.

% puxar o link com o anterior, falar que tweet se assemelha ao segundo nivel,
% comecar exemplos de referencias de SA em twitter e outras redes
A Análise de Sentimento aplicada a redes sociais, em especial ao Twitter, se
assemelha à classificação de sentimento de sentenças.
Entretanto, o perfil das mensagens que circulam nas mídias sociais apresenta
peculiaridades quando comparado a meios convencionais de comunicação escrita.
Por se tratar de um ambiente informal e de comunicação rápida, é comum encontrar
erros gramaticais e abreviações.
Similarmente, os \textit{emoticons} também tem amplo uso por serem métodos
práticos de exprimir sentimentos em poucos caracteres.
Além disso, por se tratarem de redes globais, é frequente o emprego de palavras
ou expressões de outras línguas em uma mesma mensagem.
Tais fatores são obstáculos aos classificadores de linguagem natural,
dificultando a tarefa de extração de polaridade.
A Tabela~\ref{tab:sentiment_complexity} mostra exemplos de mensagens extraídas
do Twitter que demonstram esses problemas.

% TODO alterar exemplos para nao ficar igual anterior
\begin{table}[h]
    \begin{center}
        \begin{tabular}{| l | p{10cm} |}
        \hline
        \textbf{Fator} & \textbf{\textit{Tweet}} \\ \hline
        Ironia & Recomendo chegar para dar aula e descobrir que mudaram seu horário sem avisar. \\ \hline
        Ambiguidade & Estou igualmente fascinada e enojada. \\ \hline
        Multiplicidade de idiomas & Macarrão de arroz is the new miojo. \\ \hline
        \end{tabular}
        \caption{Dificuldades encontradas na classificação de sentimento de
                 \textit{tweets}.}
        \label{tab:sentiment_complexity}
    \end{center}
\end{table}

O Twitter é uma rede social baseada em interações por mensagens curtas.
Suas principais características são a brevidade e instantaneidade das mensagens,
também chamadas \textit{tweets}.
Elas são limitadas atualmente em 280 caracteres, mas ficaram famosas pelo seu
limite anterior de 140 caracteres.
Criada em 2006, a rede conta com 139 milhões de usuários ativos
diários~\footnote{dados do relatório trimestral para
investidores~\cite{twitterreport19}} e está entre
as com maior número de usuários do mundo.

O Twitter foi responsável pela criação e popularização das \textit{hashtags},
mecanismo que funciona como marcação de palavra-chave ou tópico.
Seu funcionamento se dá pela utilização do simbolo da tralha (\#) e pela ausência
de espaço e pontuação nas construções que são formadas por mais que uma palavra.
As \textit{hashtags} funcionam como um agregador de \textit{tweets} e o site
ainda apresenta uma lista das mais populares no momento.
O sucesso dessa funcionalidade fez com que a mesma fosse posteriormente aderida
por outras mídias sociais como o Facebook e o Instagram, se tornando um
atributo marcante das redes sociais.

% faltou metadados
Além das \textit{hashtags}, as mensagens no Twitter também possibilitam a
inserção de mídias como imagens, vídeos e áudios.
Outro atributo comum em mídias sociais são as redes formadas pela conexão de
usuários.
Essas redes podem ser formadas por relações de amizade, seguidores e outras
interações entre essas pessoas ou entidades.
O re-compartilhamento de mensagem, que se dá quando um usuário divulga em seu
perfil uma mensagem criada por outra pessoa, é outro componente capaz de gerar
grafos de usuários.
A título de informação, no Twitter o re-compartilhamento também é chamado de
\textit{retweet}.

Como citado anteriormente, textos que circulam nas mídias sociais contêm
desafios adicionais em comparação a documentos tradicionalmente estudados pela
área de NLP, como artigos jornalísticos e avaliações de produto.
Apesar disso, técnicas de classificação de sentimento puramente textuais
obtiveram uma crescente melhora de desempenho, principalmente com a aplicação de
técnicas de \textit{Deep Learning}.

Um dos primeiros e mais influentes trabalhos sobre aplicação de aprendizado de
máquina para análise de sentimento foi feito por \citet{pang02}, que comparou
técnicas de classificação feitas a partir de dicionário com modelos de
\textit{Naive Bayes}, Máxima Entropia e Máquinas de Vetor Suporte (SVM).
Em seu estudo, \citet{pang02} também compara a eficácia de diferentes métodos
de representação do texto para o uso de modelos de aprendizado de máquina.

Posteriormente, inspirado nos métodos de \citet{pang02}, \citet{go09} utilizou
as mesmas técnicas para classificação de sentimento em Twitter.
A grande diferença entre os trabalhos de \citet{pang02} e \citet{go09} é que,
enquanto no primeiro caso utiliza dados de críticas de filmes, cujo coleta já
fornece uma anotação, dado que as críticas são acompanhadas de um sistema de
avaliação (entre 1 e 5 estrelas), o posterior
não contou com disponibilidade parecida e, por isso, aplicou um método de
anotação automática.

A maneira convencional de abordar bases de dados não anotadas é realizar uma
classificação manual dos dados.
Por depender de iteração humana, essa se torna a etapa mais custosa do processo
de criação de classificadores.
Além disso, a alta dinamicidade dos temas e do vocabulário presente nas redes
sociais faz com que se reduza o prazo em que a base de dados anotada é
relevante.
Considerando também que um maior volume das bases de dados, em geral, permitem
que os modelos treinados obtenham melhores resultados, a soma desses fatores torna
ineficiente o processo de anotação manual.

O processo utilizado por \citet{go09} foi criado por \citet{read05} e denominado
supervisão distante.
Este método consiste em selecionar alguma característica que tenha alta correlação
com a predição desejada e utilizá-la para anotar a base de dados.
No caso dos trabalhos citados, foram escolhidos conjuntos de \textit{emoticons}
positivos e negativos que serviram para a anotação ruidosa.
Desta forma, a restrição para construção de bases de dados passou a ser a coleta
de \textit{tweets} e de recursos computacionais.
Em \citet{go09}, a base de treinamento, que foi anotada por supervisão distante,
conteve 800 mil \textit{tweets}.
Para fins de validação do modelo, ainda foi necessário elaborar uma base de
dados anotada manualmente, sendo esta de apenas 359 \textit{tweets}.

\begin{table}[h]
    \begin{center}
        \begin{tabular}{| c c |}
        \hline
        \textbf{Emoticons Positivo} & \textbf{Emoticons Negativos} \\ \hline
        :)&:( \\
        :-)&:-(\\
        : )&: (\\
        :D& \\
        =)& \\ \hline
        \end{tabular}
        \caption{Emoticons selecionados por \citet{go09} para supervisão
        distante de \textit{tweets}.}
        \label{tab:supervision_emoticons}
    \end{center}
\end{table}

\begin{table}[h]
    \begin{center}
        \begin{tabular}{| p{10cm} | c |}
        \hline
        \textbf{\textit{Tweet}} & \textbf{Sentimento} \\ \hline
        Adoro aquelas amizades que alinham em tudo mesmo em cima da hora :) & Positivo \\ \hline
        Fico mais triste ainda porque tenho android :( & Negativo\\ \hline
        To com um mal estar tão grande no corpo desde ontem :-( zero forças & Negativo\\ \hline
        Muito feliz com minha nova tattoo :D & Positivo \\ \hline
        \end{tabular}
        \caption{Exemplos de \textit{tweets} anotados por supervisão distante
        com emoticons selecionados por \citet{go09}.}
        \label{tab:supervision_tweets}
    \end{center}
\end{table}

Entre as desvantagens dessa prática está o fato de o conjunto de
\textit{emoticons} selecionados para anotação ruidosa precisar ser removido
dos dados de treinamento do modelo para não introduzir viés na classificação.
A qualidade da supervisão distante também depende da seleção dos
\textit{emoticons}.
Ademais, não se pode descartar que é possível que uma classe contenha subclasses que
não sejam correlacionadas com a característica escolhida para executar a
supervisão distante.
Por exemplo, dentro do conjunto de \textit{tweets} positivos, a subclasse de
\textit{tweets} irônicos positivos pode não conter \textit{emoticons} e assim
não estar presente no conjunto de treinamento.
Mesmo com essas limitações, a anotação automática foi fundamental para alavancar
a performance dos classificadores de sentimento de redes sociais.

% algumas referencias em Joint Segmentation and Classification Framework for Sentence Level Sentiment Classification

A forma de se transformar o corpus textual em números foi objeto de muitos
estudos.
Como alternativa ao \textit{bag-of-words}, \citet{wang12} apresentaram um método
de utilizar pesos obtidos a partir do treinamento de um modelo de
\textit{Naïve Bayes} como entrada de um classificador de SVM.
Por sua vez, \citet{paltoglou10} estudaram variações de técnicas de Recuperação
de Informação com a mesma finalidade.
Outro eixo importante foram as representações densas, ou de baixa
dimensionalidade.
Os principais trabalhos relacionados a essas representações foram feitos por
\citet{mikolov13}, conhecido como \textit{Word2Vec}, \citet{pennington14},
denominado \textit{GloVe} e mais recentemente \textit{ELMo}, produzido por
\citet{peters18}, e \textit{BERT}, de \citet{devlin18}.

A disposição de grandes bases de dados e as representações densas foram a
condição \textit{sine qua non} para viabilizar a aplicação de modelos de
\textit{Deep Learning} em análise de sentimento.
O Aprendizado Profundo foi responsável por romper barreiras de performance
de aprendizado de máquina em diversas áreas~\cite{lecun15}, como classificação
de imagem~\cite{krizhevsky12}, reconhecimento de fala~\cite{hinton12a} e detecção
de doenças~\cite{esteva17}.
O processamento de linguagem natural foi um dos campos mais impactados por esse
conjunto de técnicas, seja na realização de traduções~\cite{vaswani17},
na correção de erros gramaticais~\cite{ge18}, no reconhecimento de
entidades~\cite{akbik18}, na criação de resumos~\cite{wu18}, ou em outras
tarefas.
Dentre as aplicações de \textit{Deep Learning} em análise de sentimento,
temos \citet{kim14}, com a utilização de redes neurais convolucionais para
classificação de sentenças, \citet{zhou16}, que, por sua vez, demonstraram o
uso de redes LSTM baseadas em redes recorrentes e \citet{socher13}, que aplicam um
modelo recursivo.

Entretanto, a mistura entre diferentes modalidades de comunicação (texto, imagem,
video, etc.) gera uma dificuldade adicional para análise das mensagens.
O texto presente em um \textit{tweet} que possuí uma imagem pode fazer
interlocução com a mesma.
Uma mensagem enviada por um usuário pode ter sentido oposto a mesma mensagem
quando comunicada por alguém que pertença a um grupo opositor.
Nesses casos, a análise do conteúdo textual por si só é incapaz de captar a
essência da mensagem.
Dá-se, assim, a necessidade de abordagens multimodais para qualquer tipo de
análise de redes sociais.

% parecido com intro mas acho ok
Como a interação entre usuários é um dos principais componentes das mídias, esse
se torna um objetivo natural para se estudar em complemento ao texto.
Para explorar esse componente, é possível aplicar ferramentas desenvolvidas pela
área de pesquisa de Redes Complexas, também chamada de Ciência de Redes.
Assim como o Aprendizado de Máquina, o ramo de estudo de Redes Complexas
apresenta um grande crescimento nos últimos anos, dado à variedade de sistemas em
que suas análises e modelos são aplicados com efetividade~\cite{albert02}.
Entre os exemplos de sucesso estão a identificação de doenças~\cite{barabasi11},
a predição de propagação de epidemias~\cite{hufnagel04}, o estudo de robustez de
redes de roteadores~\cite{albert00} e a identificação de operadores de lavagem de
dinheiro~\cite{colladon17}.

% referencia com analise de sentimento, talvez na enmlp? la tem varios trabalhos de GCNN
No caso da aplicação dessas técnicas em redes sociais, podemos citar
\citet{ratkiewicz11}.
Eles utilizaram características da rede formada por usuários do Twitter que
interagiram com \textit{tweets} sobre as campanhas eleitorais de 2010 dos
Estados Unidos, visando a detectar a propagação orquestrada de conteúdo.
Já \citet{varol18} utilizaram propriedades como densidade da rede e coeficiente de
clusterização em conjunto com outros atributos não relacionados aos grafos,
buscando a detecção de contas de usuário robôs.
\citet{backstrom11} aplicaram passeios aleatórios com pesos definidos por
atributos da rede de amizades para sugerir conexão entre dois usuários de uma
rede social.
\citet{qiu18}, por sua vez, desenvolveram uma técnica de representação de nós
baseada em redes neurais convolucionais para prever a influência de usuários em
redes sociais.
Os exemplos citados reforçam que ferramentas de Ciência de Redes são capazes de
acessar informações codificadas pela rede de usuários de mídias sociais e que essa
informação é útil para diversas aplicações.

% fazer paragrafo concluindo capitulo?

% referencias:
% Like It or Not A Survey of Twitter Sentiment Analysis Methods (cap 3.4)
% Graph-based Text Classification: Learn from Your Neighbors
% Topic Sentiment Analysis in Twitter: A Graph-based Hashtag Sentiment Classification Approach
% Twitter polarity classification with label propagation over lexical links and the follower graph
% Target-dependent Twitter Sentiment Classification

% Levantamento das tarefas envolvidas em analise de sentimento (pensando em nao escrever isso)
% - Opinions, Sentiment, and Emotion in Text
% - Many facets of Sentiment Analysis (A practical guide to sentiment analysis)
% - sentiment analysis and opinion mining (liu)
